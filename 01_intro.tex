In the last few years, there has been significant growth in grid-connected distributed energy resources (DERs) leading to an increased deployment of distributed generation (DG) and recently more distributed storage (DS) systems. Companies have started to heavily invest in the competitive energy storage system market by taking advantage of the decreasing costs of energy storage. Although a significant amount of DG and DS are being added to the distribution grid, need to improve their control systems for seamless integration to the grid is still there. In the US, most DG and DS systems are deployed to either help reduce the metered load through net-metering programs or to sell power to the utility through power purchase agreements (PPAs). The potential of DG combined with DS is not fully utilized under these pricing schemes due to the lack of proper control schemes. In order to maximize the use of available DERs, a state-of-the-art energy management solution is a necessity for our future smart grid. Such an energy management solution will be able to dynamically optimize the use of all the available DS with the objective of serving the load in the most economical and safe way possible. This will benefit both utility companies and regular consumers. Due to the constraints and intermittent nature of some DG systems, such as wind and solar, the optimum management of DS combined with a DG is a difficult problem to solve. The most common approaches found in the current literature are as follows. Some researchers formulate the problem as a linear programming (LP) or mixed integer linear programming (MILP) model \cite{lp73, lp74, lp75}. Authors in \cite{pso80, pso81} present an energy management solution based on particle swarm optimization for a microgrid containing wind turbines and energy storage (ES) system. Other researchers propose crow-search and ant colony optimization models to solve the energy management problem for local microgrids as seen in \cite{csa87} and \cite{aco84}. There have also been model predictive control (MPC) based approaches for managing ES in microgrid settings as seen in \cite{energymanajaboulay,mpcmorstyn}.  Researchers in \cite{ga76, ga77} have also proposed genetic algorithm based solutions to optimize the ES operation in a microgrid. One clear disadvantage of these proposed models is that most of these approaches only consider the current status of the system and ignore some critical factors like energy tariff, forecasted load, and forecasted generation profiles.  These information can be used to find an optimum solution based on both current and probable future states of the system as opposed to a solution relying only on the currently available data. Off-line day ahead planning models have also been proposed in the literature. In these methods, available predicted data is used to optimize the scheduling of the ES based on Monte Carlo simulations \cite{6872821,7010943,6839110}. These solutions are very computationally intensive and require a lot of time for planning the day ahead. The computational complexity makes them unsuitable for real-time implementation. Also, as they are based on off-line calculations and rely vastly on the accuracy of the predictions.

From the discussion thus far, it is evident that there is a need for a real-time ESM solution that can optimize the long-term operating costs of a system containing DG and an energy storage (ES) system. This paper presents an optimum real-time control strategy for such systems. The proposed control strategy takes into account the present and forecasted states of the system, together with the real-time price (RTP). The rest of the paper is organized as follows. Section II discusses how the ESM problem is formulated as a graph search problem. Section III introduces the A* graph search algorithm used to find the optimal solution and how it is used to search for the optimum path of energy decisions. Section IV discusses the test system used in simulation to validate the performance of the graph search based ESM algorithm. Section V discusses the results obtained from the offline simulation and section VI presents the test results obtained from the real-time simulation. Section VII presents the conclusions.